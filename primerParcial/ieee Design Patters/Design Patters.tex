\documentclass[conference]{IEEEtran}
\begin{document}

\title{Common Design Patterns in Mobile Applications}

\author{Brayan Javier Alvarez Martinez, 
0322103669@ut-tijuana.edu.mx}
\maketitle

\begin{abstract}
Design patterns are proven solutions to common problems in software development. In the context of mobile applications, design patterns play a crucial role in creating intuitive and efficient user experiences. This document examines various common design patterns in mobile applications and provides examples of their application in the design of an application for end users.
\end{abstract}

\section{Introduction}
With the exponential growth in the use of mobile devices, creating effective mobile applications has become increasingly important. Design patterns offer structured and proven solutions to recurrent challenges in mobile application design, helping developers create consistent and attractive user interfaces. This document explores the main design patterns in mobile applications and provides practical examples of their implementation.

\section{What is design pattenrs?}
Design patterns in mobile interfaces are proven and effective solutions to common challenges faced by mobile application designers. These patterns are like recipes that designers can follow to create a consistent and appealing user experience in their applications. By employing design patterns in mobile interfaces, designers can improve the usability of a mobile application and, ultimately, user satisfaction.
\section{Objective of Mobile App Design Patterns}
The main focus of design patterns in mobile interfaces is to improve the usability of mobile applications. Usability refers to how easily a user can interact with an application and perform tasks. When an application is easy to use, users are more satisfied and are more likely to use it regularly. This aspect is particularly crucial in the competitive world of mobile applications and social networks, where users have numerous options and show little tolerance for frustration.

\section{Common Design Patterns in Mobile Applications}

\subsection{Hamburger Menu}

\subsubsection{Description}
This pattern allows hiding navigation options behind a hamburger icon, maximizing screen space to display primary content.

\subsubsection{Example Application}
A news application that uses a hamburger menu to access different sections, such as breaking news, sports, and entertainment.

\subsection{Tab Bar}

\subsubsection{Description}
The tab bar provides direct access to the main sections of the application, allowing users to quickly switch between different views.

\subsubsection{Example Application}
A social media application that uses tabs to navigate between the news feed, notifications, and direct messages.

\subsection{Gesture-Based Navigation}

\subsubsection{Description}
This pattern allows users to interact with the application using touch gestures, such as swiping, pinching, and tapping.

\subsubsection{Example Application}
A mapping application that allows users to zoom in by pinching the screen and scroll by swiping their finger.

\section{Conclusions}
Design patterns play a fundamental role in creating consistent and efficient user experiences in mobile applications. By understanding and effectively applying these patterns, developers can create mobile applications that are intuitive, attractive, and easy to use for end users.

\begin{thebibliography}{1}
\bibitem{patterns}
Javier Cuello, José Vittone. (2013). \textit{Designing Mobile Apps}. Iteration and ways to hold the phone.

\bibitem{patterns}
Mateusz Grzesiukiewicz. (2018). \textit{Hands-On Design Patterns with React Native}. Styling patterns.
\end{thebibliography}

\end{document}
