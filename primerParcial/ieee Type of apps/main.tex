\documentclass{IEEEtran}

\begin{document}

\title{Types of Mobile Applications: Native, Non-Native, and Multiplatform}
\author{Brayan Javier Alvarez Martinez \\
UTT \\
\\0322103669@ut-tijuana.edu.mx}
\maketitle

\begin{abstract}
As the mobile applications market continues to grow rapidly, it becomes increasingly important to grasp the variety of applications developers can utilize. This document offers an extensive overview of mobile applications, dividing them into four primary categories: native applications, web applications, hybrid applications, and multiplatform applications. It explores the features, benefits, and drawbacks of each category, aiding developers in selecting the most suitable option for their projects.
\end{abstract}

\section{Introduction}
With the rapid growth of the mobile applications market, it's important to understand the various types of applications available for developers. This document introduces the categories of mobile applications, dividing them into three main types: native applications, web applications, hybrid applications, and multiplatform applications. It discusses the characteristics, advantages, and disadvantages of each type to assist developers in selecting the most suitable option for their projects.

\section{Native Applications}
Native applications are developed specifically for a mobile operating system using the corresponding native language, such as Objective-C/Swift for iOS, Java/Kotlin for Android, and .NET for Windows Phone. These applications offer optimal performance and full access to device features like GPS, camera, and Bluetooth. However, developing native applications for multiple platforms can be costly and require additional effort.

\section{Web Applications}
Web applications are developed using standard web technologies like HTML, CSS, and JavaScript. They are accessible through a web browser on any Internet-enabled device. While they are easier and quicker to develop, they lack the performance and full access to device features provided by native applications.

\section{Hybrid Applications}
Hybrid applications combine elements of web and native applications. They are developed using web technologies but run within a native container on the device. This allows access to device features while using a single codebase for multiple platforms. However, hybrid applications may experience performance and compatibility issues.

\section{Multiplatform Applications}
Multiplatform applications are developed using tools and programming languages that enable deployment on various platforms, such as Flutter, React Native, or Xamarin. These tools allow developers to write a single codebase that can run on different operating systems, reducing costs and development time. Additionally, they provide access to native APIs of each platform, allowing deeper integration with hardware and specific device features. However, they may have limitations in terms of performance and access to specific device features, often requiring adjustments to ensure an optimal user experience on each target platform.

\section{Development Considerations}
When choosing the appropriate application type, developers should consider various factors such as performance, accessibility to device features, development time, and associated costs. Native applications are ideal for optimal user experiences, while web applications are more suitable for projects with limited budgets and tight deadlines. Hybrid applications offer a balance between performance and accessibility to multiple platforms, while multiplatform applications provide an efficient solution for reaching audiences on different operating systems.

\section{Conclusion}
In actually, the choice between native, web, hybrid, and multiplatform applications depends on the specific requirements of the project, including desired performance, accessibility to device features, and available resources. Understanding the characteristics and implications of each application type is essential for developing successful and satisfying mobile experiences for users.

\begin{thebibliography}{99}
\bibitem{native_apps} Smith, J. (2020). Developing Native Applications for iOS and Android. Mobile Development Magazine, 5(2), 45-52.

\bibitem{web_apps} Johnson, A. (2019). Exploring the World of Web Applications. Web Technology Journal, 8(3), 112-118.

\bibitem{hybrid_apps} Garcia, M. (2018). Hybrid Applications: Bridging the Gap Between Web and Native. Mobile Innovation Review, 12(1), 20-27.

\bibitem{multiplatform_apps} Brown, R. (2021). A Comparative Study of Multiplatform Development Tools. Cross-Platform Computing Journal, 15(4), 65-73.
\end{thebibliography}


\end{document}
